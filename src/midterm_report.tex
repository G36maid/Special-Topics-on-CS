%%  請將 Typeset 改為 XeLaTeX %%
%%  左上角  Menu  → Compiler %%

\documentclass[a4paper,12pt]{article}
\usepackage[margin=2cm]{geometry} % 設定頁面邊界
% 字體
\usepackage{fontspec} % 設定字體
\usepackage{xeCJK} % 讓中英文字體分開設置
\setmainfont{Liberation Serif} % 設定英文為 Liberation Serif 字體
\setCJKmainfont{TW-MOE-Std-Kai} % 設定中文為標楷體

\XeTeXlinebreaklocale "zh" % 中文自動換行
\XeTeXlinebreakskip = 0pt plus 1pt

% 設定章節標題格式,置左粗體
\usepackage{titlesec}
\titleformat{\section}{\fontsize{12pt}{12pt}\selectfont\bfseries}{\thesection}{0em}{}
\titlespacing*{\section}{0pt}{*2}{*1.5}

% 段落間距
\setlength{\parskip}{0.5em}

% 開始文件
\begin{document}

% 封面
\begin{titlepage}
\begin{center}
\vspace*{2cm}
{\fontsize{16pt}{16pt}\selectfont\bfseries 國立臺灣師範大學 資訊工程學系}\\[1cm]
{\fontsize{16pt}{16pt}\selectfont\bfseries 114 資訊專題研究(一)期中書面報告}\\[4cm]
{\fontsize{16pt}{16pt}\selectfont\bfseries 運用 Btrfs 寫入時複製機制加速 Rust 建置快取之研究}\\[1cm]
{\fontsize{16pt}{16pt}\selectfont\bfseries A Study on Accelerating Rust Build Caching with Btrfs Copy-on-Write Mechanism}\\[9cm]
{\fontsize{12pt}{12pt}\selectfont 指導教授 紀博文 教授}\\[0.5cm]
{\fontsize{12pt}{12pt}\selectfont 學生 鍾詠傑 撰}\\[0.5cm]
{\fontsize{12pt}{12pt}\selectfont 中華民國 114 年 11 月}
\end{center}
\end{titlepage}

% 報告內容
\pagenumbering{arabic}

% 摘要
\section*{摘要}
本研究旨在探討運用 Btrfs 檔案系統的寫入時複製(Copy-on-Write, CoW)特性,以加速 Rust 語言的編譯建置快取流程。Rust 專案的編譯時間是開發流程中的主要瓶頸,而現有快取方案在處理頻繁的分支切換時效率有限。本研究提出一個基於 Btrfs 快照(snapshot)的快取框架,期望透過檔案系統層級的操作,實現近乎即時的快取還原,並降低儲存空間佔用。研究將分析此方法的技術可行性,設計一個概念性的快取架構,並規劃效能驗證藍圖。此框架的核心概念是透過 Btrfs 快照,以原子性的方式將整個 \texttt{target} 目錄切換至一個已知狀態,從而保存 Rust 的增量編譯資料庫。相較之下,\texttt{sccache} 等編譯器層級的工具無法對整個建置目錄進行原子性快照還原。

\section*{研究動機與研究問題}
Rust 語言以其安全性與高效能獲得廣泛應用,但其複雜的編譯過程常導致開發者需耗費大量時間等待專案建置,嚴重影響開發迭代效率。為解決此問題,Rust 編譯器內建了增量編譯機制 \cite{rust_incremental_blog},而社群也開發了如 \texttt{sccache} \cite{sccache_repo} 等外部快取工具。然而,這些方案在處理頻繁的程式碼分支切換時仍存在瓶頸。特別是當開發者執行 \texttt{git checkout} 切換分支時,原始碼檔案的時間戳會改變,常導致 Rust 原生的增量編譯快取大量失效,引發不必要的重新編譯。

Btrfs 檔案系統提供了原生的寫入時複製(CoW)功能,包含 reflink(檔案層級的 CoW 連結)與快照(子卷層級的即時備份)。Docker 已在生產環境中運用 Btrfs 驅動程式,將映像檔分層對應到 Btrfs 子磁區,並將容器的可寫狀態對應到 Btrfs 快照 \cite{docker_btrfs_driver},這為本研究提供了技術上的參考。我們觀察到,Rust 編譯過程產生的 \texttt{target} 目錄結構龐大且內容相似度高,特別是其增量編譯資料庫對於實現快速重新建置至關重要 \cite{rust_incremental_detail}。這正符合 Btrfs CoW 機制的應用場景。

基於以上觀察,本研究希望回答以下核心研究問題:
\begin{enumerate}
    \item 如何利用 Btrfs 的快照機制,設計一個高效能、低儲存佔用的 Rust 編譯快取框架?
    \item 相較於在 ext4 等傳統檔案系統上使用標準檔案複製,Btrfs 的 CoW 操作在快取建置與還原速度上能帶來多大的效能提升?
    \item 此框架應如何與 Rust 原生的增量編譯及 \texttt{sccache} 等外部工具整合,以形成一個更全面的多層次快取體系?
\end{enumerate}

\section*{初步文獻探討}
現有的 Rust 編譯快取方案主要分為不同層次。第一層是 Rust 編譯器原生的增量編譯機制,它採用精密的演算法來追蹤相依性,並將中間結果快取於 \texttt{target/incremental} 目錄中 \cite{rust_incremental_blog}。第二層是 \texttt{sccache} 等編譯器包裝器工具,它透過攔截編譯指令,對編譯單元的輸入進行雜湊運算來產生快取鍵,並支援本地或遠端儲存 \cite{sccache_repo}。

這兩種快取機制運作於不同的抽象層級:Rust 增量編譯最為精細,理解語言內部語義與相依圖;\texttt{sccache} 運作於編譯器調用層級,將 \texttt{rustc} 視為黑盒子。關鍵差異在於,當開發者切換 Git 分支時,第一層增量快取會因時間戳改變而失效,第二層 \texttt{sccache} 雖可快速提供編譯產物,但無法保存增量快取內部狀態。本研究探討一個「第零層」的快取機制,以檔案系統快照保存整個 \texttt{target} 目錄的狀態,從而完整保護第一層快取在跨分支切換時的完整性。

\section*{初步研究方法}
本研究提出一個名為「快照交換模型」(Snapshot Swap Model)的概念性快取框架。其核心設計是將 Rust 專案的 \texttt{target} 編譯輸出目錄存放於一個獨立的 Btrfs 子卷中,並透過快照管理其不同的建置狀態。

\textbf{工作流程設計:} 專案的原始碼與 \texttt{target} 目錄必須位於同一個 Btrfs 檔案系統上。(1) \textbf{基準快照建立:} 在主分支上成功完成一次全新建置後,為 \texttt{target} 子磁區建立一個唯讀快照。(2) \textbf{開發開始:} 當開發者切換到功能分支時,從最相關的基準快照建立一個新的可讀寫快照,作為當前活躍的 \texttt{target} 目錄。(3) \textbf{增量建置:} 所有後續的 \texttt{cargo build} 指令都在這個可讀寫的快照中進行。(4) \textbf{情境切換:} 當執行 \texttt{git checkout} 時,透過腳本將活躍的 \texttt{target} 目錄原子性地切換回對應分支的快照。

\textbf{概念驗證計畫:} 為驗證此方法,我們將設計一個概念驗證腳本,在一個中等規模的 Rust 專案上進行基準測試。測試將比較在 Btrfs 與 ext4 兩種檔案系統上,執行「清除、切換分支、重新編譯」等典型開發場景所需的時間。效能評估將基於以下指標:(1) \textbf{主要指標:重新建置時間。} 測量在兩個具備顯著差異的分支之間執行 \texttt{git checkout} 後,重新編譯專案所需的時間。預期此方法相較於傳統快取方案,在時間上能有顯著縮短。(2) \textbf{次要指標:} 記錄快照建立與還原操作的耗時、多次建置循環後的磁碟空間使用率,以及快照管理腳本本身的效能開銷。

本研究所提出的方法,其預期效益在於縮短開發情境切換所需的時間、確保快取狀態的一致性,並減少本地開發的 I/O 開銷。然而,此方法亦存在待評估的挑戰,包括對特定檔案系統(Btrfs)的依賴性、快照管理腳本的實作複雜度、長期使用下可能的儲存空間佔用,以及 Btrfs 在特定 I/O 負載下的效能表現 \cite{phoronix_btrfs}。本研究的概念驗證結果,將作為評估此方法實用性的依據。本研究的適用範圍目前限於採用 Btrfs 檔案系統的 Linux 開發環境。

\begin{thebibliography}{9}
\bibitem{docker_btrfs_driver}
Docker Documentation. \textit{BTRFS storage driver}. [Online]. Available: \texttt{https://docs.docker.com/engine/storage/drivers/btrfs-driver/}

\bibitem{rust_incremental_blog}
The Rust Programming Language Blog. (2016). \textit{Incremental Compilation}. [Online]. Available: \texttt{https://blog.rust-lang.org/2016/09/08/incremental.html}

\bibitem{rust_incremental_detail}
Rust Compiler Development Guide. \textit{Incremental compilation in detail}. [Online]. Available: \texttt{https://rustc-dev-guide.rust-lang.org/queries/incremental-compilation-in-detail.html}

\bibitem{sccache_repo}
Mozilla. \textit{mozilla/sccache}. [Online]. GitHub. Available: \texttt{https://github.com/mozilla/sccache}

\bibitem{phoronix_btrfs}
Phoronix. (2024). \textit{Bcachefs, Btrfs, EXT4, F2FS \& XFS File-System Performance On Linux 6.15}. [Online]. Available: \texttt{https://www.phoronix.com/review/linux-615-filesystems}

\end{thebibliography}

\end{document}
