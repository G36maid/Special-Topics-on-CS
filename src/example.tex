%%  請將 Typeset 改為 XeLaTeX %%
%%  左上角  Menu  → Compiler %%

\documentclass[a4paper,12pt]{article}
\usepackage[margin=2cm]{geometry} % 設定頁面邊界
% 字體
\usepackage{fontspec} % 設定字體
\usepackage{xeCJK} % 讓中英文字體分開設置
\setmainfont{Liberation Serif} % 設定英文為 Liberation Serif 字體
\setCJKmainfont{TW-MOE-Std-Kai} % 設定中文為標楷體

\XeTeXlinebreaklocale "zh" % 中文自動換行
\XeTeXlinebreakskip = 0pt plus 1pt

% 設定章節標題格式,置左粗體
\usepackage{titlesec}
\titleformat{\section}{\fontsize{12pt}{12pt}\selectfont\bfseries}{\thesection}{0em}{}
\titlespacing*{\section}{0pt}{*2}{*1.5}

% 段落間距
\setlength{\parskip}{0.5em}

% 開始文件
\begin{document}

% 封面
\begin{titlepage}
\begin{center}
\vspace*{2cm}
{\fontsize{16pt}{16pt}\selectfont\bfseries 國立臺灣師範大學 資訊工程學系}\\[1cm]
{\fontsize{16pt}{16pt}\selectfont\bfseries 114 資訊專題研究(一)期中書面報告}\\[4cm]
{\fontsize{16pt}{16pt}\selectfont\bfseries 中文主題名稱}\\[1cm]
{\fontsize{16pt}{16pt}\selectfont\bfseries 英文主題名稱}\\[9cm]
{\fontsize{12pt}{12pt}\selectfont 指導教授 OOO 教授}\\[0.5cm]
{\fontsize{12pt}{12pt}\selectfont 學生 OOO 撰}\\[0.5cm]
{\fontsize{12pt}{12pt}\selectfont 中華民國 114 年 11 月}
\end{center}
\end{titlepage}

% 報告內容
\pagenumbering{arabic}

% 摘要
\section*{摘要}
(請在此處填寫您的摘要內容。摘要應簡潔說明研究的核心內容、方法與預期貢獻。)

% 研究動機與研究問題
\section*{研究動機與研究問題}
(請在此處填寫您的研究動機與研究問題。說明為什麼這個主題重要,以及您希望透過本研究回答的具體問題。)

% 初步文獻探討
\section*{初步文獻探討}
(請在此處填寫您的初步文獻探討。簡要回顧相關領域的現有研究,並說明您的研究與它們的關係。)

% 初步研究方法
\section*{初步研究方法}
(請在此處填寫您的初步研究方法。描述您計畫如何進行研究、收集資料與分析,以回答您的研究問題。)

\end{document}
